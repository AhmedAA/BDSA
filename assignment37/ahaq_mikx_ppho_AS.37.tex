\documentclass[10pt]{report}
\usepackage[utf8]{inputenc}
\usepackage[T1]{fontenc}
\usepackage[english]{babel}
\usepackage{fourier} % Nicenesss font
\usepackage{amsfonts,amsmath}
\usepackage[pdftex]{color,graphicx}
\usepackage{pdfpages}
\usepackage{fancyhdr}
\usepackage{textcomp}
\usepackage{lmodern}
\usepackage{xcolor}
\usepackage{multicol}
\usepackage{soul}
\usepackage{algorithmic}
\usepackage{wrapfig}
\usepackage{float}
\usepackage{mdwlist}
\usepackage{pgfgantt}
\usepackage{a4wide}
\usepackage{tabularx}
\usepackage{booktabs}

\newcommand{\tabitem}{~~\llap{\textbullet}~~}

\usepackage{hyperref}
\usepackage{lscape}

\usepackage{sectsty}
\allsectionsfont{\centering \normalfont\scshape}
\renewcommand{\thesection}{\arabic{section}}

\numberwithin{equation}{section} % Number equations within sections (i.e. 1.1, 1.2, 2.1, 2.2 instead of 1, 2, 3, 4)
\numberwithin{figure}{section} % Number figures within sections (i.e. 1.1, 1.2, 2.1, 2.2 instead of 1, 2, 3, 4)
\numberwithin{table}{section} % Number tables within sections (i.e. 1.1, 1.2, 2.1, 2.2 instead of 1, 2, 3, 4)

%\usepackage{hyperref}
\pagestyle{fancy}
\newcommand{\HRule}{\rule{\linewidth}{0.5mm}}

%If in need of a header for the document, uncomment this and add desired text!
%\fancyhead[LO,LE]{}
%\fancyhead[RO,RE]{}
%

%Macro foo
\newcommand{\method}[3]{ \label{method:#2}
    {\vspace{10pt} \noindent \textbf{#1} \textit{#2}(#3)} :
}
\newcommand{\argument}[2]{{\textbf{#1} #2}}


%%%%%%%%%%%% END OF PREAMBLE %%%%%%%%%%%%%%%%%%%%%%%%%%%%%%%%%%%%
\begin{document}
\begin{titlepage}

\begin{center}

\textsc{\LARGE BDSA}\\[1.5cm]

\textsc{\Large Requirement Analysis Document}\\[0.5cm]

\HRule \\[0.4cm]

{ \bfseries Assignment \#37 \\[0.5cm] 
    {\small \today}} \\[0.7cm]

\HRule \\ [6.5cm]

% Author and supervisor
\begin{minipage}{0.5\textwidth}
\begin{flushleft} \large
Ahmed Al Aqtash (ahaq)\\
\textit{ahmed.aqtash@gmail.com}\\
Mikkel Åxman (mikx)\\
\textit{mikx@itu.dk}\\
Phillip Phoelich (ppho)\\
\textit{ppho@itu.dk}\\

\vfill 
\end{flushleft}
\end{minipage}
% Bottom of the page front page

\end{center}

\end{titlepage}
\clearpage
\begin{table}[h]
\begin{tabularx}{\textwidth}{l l X l}
\textbf{Version} & \textbf{Date} & \textbf{Description} & \textbf{Authors} \\ \midrule
1.0     & 9/9-2014 & First version of the document & ahaq, mikx, ppho  \\
        &          &                               & \\
        &          &                               & \\
\end{tabularx}
\end{table}

\clearpage
\section{Actors}
The actors identified in this section have been selected based on both users and
back end system. The intermediary layers (if any) are also here
\begin{itemize}
\item \textbf{User} - The end user of the system, who wants to add calendars or
  events to his client.
\item \textbf{Client/Front end} - The client program that the user uses to view
  his calendar.
\item \textbf{Provider} - A provider is a user that provides calendar data to be
  used in the system. A user can subscribe to a calendar from a provider.
\end{itemize}

\section{Scenarios}
\begin{table}[H]
\noindent \textbf{Add calendar}\\
\begin{tabularx}{\textwidth}{l X}
\midrule
\textit{Scenario name} & Add local calendar \\ \midrule
\textit{Participating actor} & \underline{Bob:User} \\ \midrule
\textit{Flow of events} & \tabitem Bob opens his client, and clicks the add
                                       calendar button \\
                                       & \tabitem The client presents Bob with
                                       an option to add a local calendar, or a
                                       calendar from a network location (URL). \\
                                       & \tabitem Bob adds a local calendar, and
                                       names it accordingly. \\
                                       & \tabitem Bob can now view his newly
                                       added calendar. \\
                                       \midrule
\end{tabularx}
\end{table}

\begin{table}[H]
\noindent \textbf{Add event}\\
\begin{tabularx}{\textwidth}{l X}
\midrule
\textit{Scenario name} & Add local event \\ \midrule
\textit{Participating actor} & \underline{Bob:User} \\ \midrule
\textit{Flow of events} & \tabitem Bob opens his client, and clicks the add
                                       event button \\
                                       & \tabitem The client presents Bob with
                                       a box to fill out information regarding
                                       the event, and the calendar to which Bob
                                       would like to submit the event.  \\
                                       & \tabitem Bob clicks add event.  \\
                                       & \tabitem Bob can now view his newly
                                       added event. \\
                                       \midrule
\end{tabularx}
\end{table}

\section{Use cases}
The list of use cases here is not exhaustive, but gives a general idea of the
basic functionality the system should have. As we progress, it will be further
developed. There is an overlap between both the user and the back end system on
some of the cases, this is going to be clearly outlined in the diagram.
\begin{itemize}
\item \textbf{Add calendar}
\item \textbf{Add event}
\item \textbf{Add calendar from URL}
\item \textbf{Remove calendar }
\item \textbf{Remove event}
\item \textbf{Unsubscribe from calendar}
\item \textbf{Switch view day/week/month}
\item \textbf{Select which calendars to view}
\end{itemize}

Subscribing to a calendar is a sub-case of adding af calendar. Unsubscribing is
thus also a sub-case of removing a calendar.

\subsection{Diagram}

\section{Non-trivial use cases}
\begin{table}[H]
\noindent \textbf{Subscribe calendar}\\
\begin{tabularx}{\textwidth}{l X}
\midrule
\textit{Use case name} & SubscribeCalendar \\ \midrule
\textit{Participating actor} & Initiated by User \\ \midrule
\textit{Flow of events} & \tabitem User clicks the add calendar button\\
                                       & \tabitem User selects a subscription
                                       based calendar from the add calendar
                                       event\\
                                       & \tabitem User adds URL\\
                                       & \tabitem User can now view added calendar events
                                       in client\\
                        \midrule
\textit{Entry condition} & User clicks add calendar.\\ \midrule
\textit{Exit condition} & \tabitem When user has successfully added a calendar. \\
						& \tabitem User cancels the operation.\\
                        \midrule
\end{tabularx}
\end{table}

\begin{table}[H]
\noindent \textbf{Add event}\\
\begin{tabularx}{\textwidth}{l X}
\midrule
\textit{Use case name} & AddEvent \\ \midrule
\textit{Participating actor} & Initiated by User \\ \midrule
\textit{Flow of events} & \tabitem User clicks the add event button\\
                                       & \tabitem User selects a calendar where
                                       the event will be added.\\
                                       & \tabitem User types details for event, such as
                                       name, place, note.\\
                                       & \tabitem User can now view event in selected calendar.\\
                        \midrule
\textit{Entry condition} & User clicks add event.\\ \midrule
\textit{Exit condition} & \tabitem When user has successfully added  an event. \\
						& \tabitem User cancels the operation.\\
                        \midrule
\end{tabularx}
\end{table}

\begin{table}[H]
\noindent \textbf{Remove event}\\
\begin{tabularx}{\textwidth}{l X}
\midrule
\textit{Use case name} & RemoveEvent \\ \midrule
\textit{Participating actor} & Initiated by User \\ \midrule
\textit{Flow of events} & \tabitem User clicks an event\\
                                       & \tabitem User selects remove/delete
                                       event from context menu.\\
                                       & \tabitem Users even has now been removed.\\
                        \midrule
\textit{Entry condition} & User clicks an event.\\ \midrule
\textit{Exit condition} & \tabitem When user has successfully removed an event. \\
						& \tabitem User cancels the operation.\\
                        \midrule
\end{tabularx}
\end{table}

\section{Initial Analysis Objects}

\end{document}
