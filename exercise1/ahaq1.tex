\documentclass[10pt]{report}
\usepackage[utf8]{inputenc}
\usepackage[T1]{fontenc}
\usepackage[english]{babel}
\usepackage{fourier} % Nicenesss font
\usepackage{amsfonts,amsmath}
\usepackage[pdftex]{color,graphicx}
\usepackage{pdfpages}
\usepackage{fancyhdr}
\usepackage{textcomp}
\usepackage{lmodern}
\usepackage{xcolor}
\usepackage{multicol}
\usepackage{soul}
\usepackage{algorithmic}
\usepackage{wrapfig}
\usepackage{float}
\usepackage{mdwlist}
\usepackage{pgfgantt}

\usepackage{hyperref}
\usepackage{lscape}

\usepackage{sectsty}
\allsectionsfont{\centering \normalfont\scshape}
\renewcommand{\thesection}{\arabic{section}}

\numberwithin{equation}{section} % Number equations within sections (i.e. 1.1, 1.2, 2.1, 2.2 instead of 1, 2, 3, 4)
\numberwithin{figure}{section} % Number figures within sections (i.e. 1.1, 1.2, 2.1, 2.2 instead of 1, 2, 3, 4)
\numberwithin{table}{section} % Number tables within sections (i.e. 1.1, 1.2, 2.1, 2.2 instead of 1, 2, 3, 4)

%\usepackage{hyperref}
\pagestyle{fancy}
\newcommand{\HRule}{\rule{\linewidth}{0.5mm}}

%If in need of a header for the document, uncomment this and add desired text!
%\fancyhead[LO,LE]{}
%\fancyhead[RO,RE]{}
%

%Macro foo
\newcommand{\method}[3]{ \label{method:#2}
    {\vspace{10pt} \noindent \textbf{#1} \textit{#2}(#3)} :
}
\newcommand{\argument}[2]{{\textbf{#1} #2}}


%%%%%%%%%%%% END OF PREAMBLE %%%%%%%%%%%%%%%%%%%%%%%%%%%%%%%%%%%%
\begin{document}
\begin{titlepage}

\begin{center}

\textsc{\LARGE BDSA}\\[1.5cm]

\textsc{\Large Exercise 1}\\[0.5cm]

\HRule \\[0.4cm]

{ \bfseries Assignment \#35 \\[0.5cm] 
    {\small \today}} \\[0.7cm]

\HRule \\ [6.5cm]

% Author and supervisor
\begin{minipage}{0.5\textwidth}
\begin{flushleft} \large
Ahmed Al Aqtash (ahaq)\\
\textit{ahmed.aqtash@gmail.com}\\
[0.2cm]

\vfill 
\end{flushleft}
\end{minipage}

% Bottom of the page front page


\end{center}

\end{titlepage}
\clearpage

\section{Question 1-2}
An advantage to using a programming language to define everything in the
development process is that provides an easier transition from design, to
actual implementation (if not eliminate the entire step between). It should
also provide more consistency in the project, since everything is defined with
the same terms.

A clear disadvantage would be that the development team is entirely limited to
programmers, since programmers are mainly the only ones who could understand
what exactly was going on. This prevents the team from actually using people 
with specific knowledge in designated areas, such as Human Computer 
Interaction experts, or visual design experts.

\section{Question 1-4}
That knowledge aqcuisition is not sequential, means that a person does not
absorb new information as it is piled on. A persons ability to learn varies,
but it is mainly in a way where the person absorbs bits, and needs
to keep refreshing what he/she has learned.

An example of this, is when a person understands a given problem and solution
after haveing studied it several times. Although the person has been filled
with the necessary information, he/she does not necessarily understand why
things work out as they do. This should be a prime example of how knowledge is
being acquired in parrallel. A specific example would be when a class is
being taught in a new subject. Students who have studied ahead, should have
and easier time grasping this new subject, than students who have not.

\section{Question 1-6}
\begin{itemize}
\item “The TicketDistributor must enable a traveler to buy weekly passes.”
    Is a functional requirement, since it has to do with a function the system
    must support.
\item “The TicketDistributor must be written in Java.” Is a non-functional
    requirement, since it has nothing to do with what the product should
    support.
\item “The TicketDistributor must be easy to use.” Is also a non-functional
    requirement. It has to do with how the users interact with the system, but
    not what the system supports.
\item “The TicketDistributor must always be available.” Is also a
    non-functional requirement. This also has to do with how users interact
    with the system.
\item “The TicketDistributor must provide a phone number to call when it
    fails." I am uncertain about this one. I would categorise it as a
    functional requirement, since it could technically be a feature of the
    system, but then again....
\end{itemize}

\section{Question 1-8}
The two first sentences talk about an account as a bank account. The third
sentence talks about a user account (application domain), and the last
sentence again mentions accounts as bank accounts.

\end{document}
